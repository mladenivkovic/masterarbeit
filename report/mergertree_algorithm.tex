%====================================================================
\section{Making and Testing Merger Trees}\label{chap:my_code}
%====================================================================

%============================================
\subsection{Making Merger Trees}
%============================================


The first step for any merger tree code is to identify plausible progenitor candidates for descendant clumps as well as descendant candidates for progenitor clumps.
In this algorithm, this is done by tracking up to a maximal number, labelled $n_{mb}$, of particles per progenitor clump.
The minimal number of tracker particles is given by the mass threshold for haloes.
The tracker particles of any halo are chosen to be the $n_{mb}$ particles with the lowest energy $E$:
\begin{align*}
	E = m \mathbf{v} ^2 + \phi(\mathbf{r})
\end{align*}
where $m$ is the particle's mass, $\mathbf{v}$ is the particle's velocity relative to the halo's bulk velocity, $\phi$ is the gravitational potential of the halo and $\mathbf{r}$ is the position of the particle.
If $E < 0$, a particle is considered to be energetically bound to the halo. 
Selecting the $n_{mb}$ particles with lowest energy $E$ thus corresponds to choosing the ``$n_{mb}$ \textbf{m}ost (tightly) \textbf{b}ound particles of the halo''.
This choice is made because the most strongly bound particles are expected to more likely remain within the clump between two snapshots.


For every clump in the current snapshot, the $n_{mb}$ tracker particles are found and written to file.
In the following output step, those files will be read in and sorted out: 
The clumps of the previous snapshot will be the progenitors of this snapshot. 
Based on in which descendant clump each progenitor's particles ended up in, progenitors and descendants are linked, i.e. possible candidates are identified this way.
 
Next, the main progenitor of each descendant and the main descendant of each progenitor need to be found. 
This search is performed iteratively.
A main progenitor-descendant pair is established when the main progenitor of a descendant is the main descendant of said progenitor. 
At every iteration, all descendant candidates of all progenitors that haven't found their match yet are checked;
The descendants without a matching progenitor however only move on to the next best progenitor candidate. 
For both descendants and progenitors, all candidates are ranked from ``best'' to ``worst'' based on the merit function \eqref{eq:merit}.
The iteration is repeated until every descendant has checked all of its progenitor candidates or found its match. 
Progenitors that haven't found a main descendant that isn't taken yet will be considered to have merged into their best fitting descendant candidate.

After the iteration, any progenitor that is considered as merged into its descendant will be recorded as a ``past merged progenitor''.
Only one, the most strongly bound, particle and the time of merging will be stored for past merged progenitors.
This particle is referred to as the ``galaxy particle'' of the merged progenitor.
Storing this data will allow to check in later, non-consecutive snapshots whether the progenitor has truly merged into its main descendant and to track orphan galaxies.

Then descendants that still haven't got a progenitor at this point will try to find one in non-consecutive past snapshots:
The particles that the descendant consists of are checked for being a galaxy particle of a past merged progenitor. 
The most strongly bound galaxy particle will be considered the main progenitor of the descendant under consideration.


Descendants that still haven't found a progenitor at this point are deemed to be newly formed.
This concludes the tree-making and the results are written to file.

Because every processing unit in the current implementation reads in all progenitor data, which unlike the current clump data doesn't change any more, no intricate and flexible communication structures like a peak communicator for the clump finder are necessary.
Simple collective MPI communications suffice.

Lastly, there is an option to remove past merged progenitors from the list once they merged into their main descendants too many snapshots ago.
By default, the algorithm will store them until the end of the simulation.
For the interested reader, a detailed description of the merger tree algorithm is given in appendix \ref{app:detailed_mergertree}.





