\begin{table}[!ht]
	\begin{center}

		{\small 
			\begin{tabular}[c]{l | p{1.8cm} | p{1.8cm} | p{1.8cm} | p{1.8cm} |}
											&	\inc\ \sad & \exc\ \sad\ & \inc\ \nosad\ & \exc\ \nosad\ \\ 
				
				\hline
				%
				total clumps	 						&	16262	& 	16262	& 	17242	&	17242 	\\
				
				%		
				max number of particles in a clump 		&	414570	& 	414570	& 	271438	&	271438 	\\
				
				%		
				median number of particles in a clump 	&	84		& 	83		& 	93		&	93 		\\
				
				%
				\hline
				%		
				average main branch length group I		&	22.980	& 	23.153	& 	20.426	&	20.527 	\\
				
				%
				average main branch length group II		&	48.653	& 	48.835	& 	48.132	&	48.642 	\\
				
				%		
				average main branch length group III	&	54.358	& 	54.715	& 	54.305	&	54.904 	\\
				
				%
				average main branch length group IV		&	54.546	& 	53.958	& 	56.110	&	56.265 	\\
				
				%
				\hline
				%		
				average number of branches group I		&	1.318	& 	1.327	& 	1.230	&	1.230 	\\
				
				%
				average number of branches group II		&	3.480	& 	3.448	& 	3.632	&	3.603 	\\
				
				%		
				average number of branches group III	&	8.466	& 	8.670	& 	8.673	&	8.661 	\\
				
				%
				average number of branches group IV		&	29.771	& 	29.457	& 	28.783	&	28.607 	\\	
				
				%
				\hline	
			\end{tabular}
		}
		\caption{
			Average data for all clumps at $z=0$ for the four halo catalogue modifying parameter pairs: whether subhalo particles are included (\inc) or excluded (\exc) in the clump mass of satellite haloes, and whether to consider particles which might wander off into another clump as bound (\nosad) or not (\sad).
			The groups I, II, III and IV are defined as clumps that contain less then 100, 100-500, 500-1000 or more than 1000 particles, respectively.
		}
		\label{tab:saddle_nosaddle}
	\end{center}	
\end{table}