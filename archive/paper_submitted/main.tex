\pdfoutput=1 %for arxiv
% Header for MNRAS template
% v3.0 released 14 May 2015
% (version numbers match those of mnras.cls)
%
% Copyright (C) Royal Astronomical Society 2015
% Authors:
% Keith T. Smith (Royal Astronomical Society)

% Change log
%
% v3.0 May 2015
%    Renamed to match the new package name
%    Version number matches mnras.cls
%    A few minor tweaks to wording
% v1.0 September 2013
%    Beta testing only - never publicly released
%    First version: a simple (ish) template for creating an MNRAS paper

%%%%%%%%%%%%%%%%%%%%%%%%%%%%%%%%%%%%%%%%%%%%%%%%%%
% Basic setup. Most papers should leave these options alone.
\documentclass[a4paper,twocolumn,fleqn,usenatbib]{mnras}
% MNRAS is set in Times font. If you don't have this installed (most LaTeX
% installations will be fine) or prefer the old Computer Modern fonts, comment
% out the following line
%\usepackage{newtxtext,newtxmath}
% Depending on your LaTeX fonts installation, you might get better results with one of these:
%\usepackage{mathptmx}
%\usepackage{txfonts}

% Use vector fonts, so it zooms properly in on-screen viewing software
% Don't change these lines unless you know what you are doing
\usepackage[T1]{fontenc}
\usepackage{ae,aecompl}
\usepackage{caption}


%%%%% AUTHORS - PLACE YOUR OWN PACKAGES HERE %%%%%

% Only include extra packages if you really need them. Common packages are:
\usepackage{graphicx}	% Including figure files
\usepackage{amsmath}	% Advanced maths commands
\usepackage{amssymb}	% Extra maths symbols

%%%%%%%%%%%%%%%%%%%%%%%%%%%%%%%%%%%%%%%%%%%%%%%%%%

%%%%% AUTHORS - PLACE YOUR OWN COMMANDS HERE %%%%%

% Please keep new commands to a minimum, and use \newcommand not \def to avoid
% overwriting existing commands. Example:
%\newcommand{\pcm}{\,cm$^{-2}$}	% per cm-squared





% Gleich mit Dach obendrauf
%\newcommand{\entspricht}{\mathrel{\widehat{=}}}
% Atanh
%\newcommand {\arctanh}{\mathrm{arctanh}}
% Acotanh
%\newcommand{\arccot}{\mathrm{arccot }}
% Limes von etwas gegen null
%\newcommand{\limz}[1]{\lim\limits_{#1 \rightarrow 0}}
%Bold font in math
%\newcommand{\bm}{\boldmath}
%\newcommand{\dps}{\displaystyle}
% e noncursive in math mode
%\newcommand{\e}{\mbox{e}}
% partial diff operator
\newcommand{\del}{\partial}
%differential d
\newcommand{\de}{\mathrm{d}}

%\newcommand{\ramses}{\textsc{ramses}}
%\newcommand{\phew}{\textsc{phew}}
%\newcommand{\dice}{\textsc{dice}}
\newcommand{\ramses}{\texttt{RAMSES}}
\newcommand{\phew}{\texttt{PHEW}}
\newcommand{\dice}{\texttt{DICE}}

\newcommand{\exc}{\texttt{exclusive}}
\newcommand{\inc}{\texttt{inclusive}}
\newcommand{\sad}{\texttt{strictly bound}}
\newcommand{\nosad}{\texttt{loosely bound}}

\newcommand{\gsmall}{\texttt{G69}}
\newcommand{\glarge}{\texttt{G100}}

\newcommand{\dt}{\texttt{dice-twobody}}
\newcommand{\ds}{\texttt{dice-levels}}
\newcommand{\cosmo}{\texttt{cosmo}}
\newcommand{\simple}{\texttt{simple}}
\newcommand{\neigh}{\texttt{neighbours}}
\newcommand{\iter}{\texttt{iter}}
\newcommand{\phewon}{\texttt{PHEW only}}
\newcommand{\msol}{M_{\sun}}











%%%%%%%%%%%%%%%%%%%%%%%%%%%%%%%%%%%%%%%%%%%%%%%%%%

%%%%%%%%%%%%%%%%%%% TITLE PAGE %%%%%%%%%%%%%%%%%%%

% Title of the paper, and the short title which is used in the headers.
% Keep the title short and informative.
\title[ACACIA]{{ACACIA}: a new method to produce on-the-fly merger trees in the RAMSES code }

% The list of authors, and the short list which is used in the headers.
% If you need two or more lines of authors, add an extra line using \newauthor
\author[Mladen Ivkovic]{
    Mladen Ivkovic,$^{1}$$^{2}$\thanks{E-mail: mladen.ivkovic@epfl.ch}
    Romain Teyssier$^{3}$
    \\
    % List of institutions
    $^{1}$Laboratoire d'Astrophysique, \'Ecole Polytechnique F\'ed\'erale de Lausanne, 1290 Versoix, Switzerland\\
    $^{2}$Observatoire de Gen\`eve, Universit\'e de Gen\`ve, Chemin Pegasi 51, 1290 Versoix, Switzerland\\
    $^{3}$Institute for Computational Science, University of Zurich, 8057 Zurich, Switzerland\\
}

% These dates will be filled out by the publisher
\date{Accepted XXX. Received YYY; in original form ZZZ}

% Enter the current year, for the copyright statements etc.
\pubyear{2021}






% Don't change these lines
\begin{document}
    \label{firstpage}
    \pagerange{\pageref{firstpage}--\pageref{lastpage}}
    \maketitle
    
    \begin{abstract}
	\section*{Abstract}
	
	The implementation of an algorithm to identify dark matter halo merger trees into the adaptive mesh refinement code \ramses\ is presented.
	The algorithm is fully parallel using MPI and works on the fly.
	It tracks dark matter substructure individually through particle IDs, thus allowing to follow the formation history of dark matter clumps up to the point where they dissolve beyond the possibility of identification as substructure.	
	Once a clump merges into another, it is still being tracked by marking the most strongly bound particle of that clump at the last snapshot where it was identified.
	This allows to check at later snapshots whether the identified merging event truly was one, or whether a misidentification by the density field gradient based clump finding algorithm might have occured.
    The influence of various definitions of substructure and the maximal number of particles tracked per clump have been tested.	
	
	With the known formation history of dark matter structure, galaxies can be introduced in a simulation containing only dark matter particles through use of a parametrised stellar-mass-to-halo-mass (SMHM) relation.
	In this work, the SMHM relation published by \cite{Behroozi} was used. 
	The obtained stellar mass functions of central galaxies from $z\sim 0 - 8$ and correlation functions at $z \sim 0$ are compared to observational data.
	Taking into account that the simulations used to obtain these mock galaxy catalogues didn't have particularly high resolutions with $512^3$ particles and that the main focus of this work is to demonstrate that a merger tree algorithm and the generation of mock galaxy catalogues can be done on the fly, the results show satisfactory agreement with observational data.

		
		
%	To demonstrate the influence of user-defined parameters of the merger tree algorithm, the results of tests similar to those of the ``sussing merger tree'' paper series \parencite{SUSSING_COMPARISON} are shown.
	

\end{abstract}
    
    % Select between one and six entries from the list of approved keywords.
    % Don't make up new ones.
    \begin{keywords}
    	methods: numerical -- galaxies: evolution -- galaxies: haloes -- dark matter
    \end{keywords}
    

     
    
    
    
%%%%%%%%%%%%%%%%%%%%%%%%%%%%%%%%%%%%%%%%%%%%%%%%%%

%%%%%%%%%%%%%%%%% BODY OF PAPER %%%%%%%%%%%%%%%%%%
    
    
    %==================================
\section{Introduction}
%==================================

Mock galaxy catalogues generated using N-body or hydrodynamical
simulations are important tools for extragalactic astronomy and
cosmology.  They are used to test current theories of galaxy
formation, to explore systematic and statistical errors in large scale
galaxy surveys and to prepare analysis codes for future dark energy
mission such as Euclid or LSST.  There is a large variety of methods
to generate such mock galaxy catalogues.  The most ambitious line of
products is based on full hydrodynamical simulations, where dark
matter, gas, and star formation are directly simulated
\citep{duboisDancingDarkGalactic2014,
  khandaiMassiveBlackIISimulationEvolution2015,
  vogelsbergerPropertiesGalaxiesReproduced2014,
  schayeEAGLEProjectSimulating2015}.  The intermediate approach is
based on semi-analytic modelling (hereafter SAM) \citep{SA-white,
  SA-durham, SA-Somerville, SA-Kaufmann,
  kangSemianalyticalModelGalaxy2005,crotonManyLivesActive2006,somervilleSemianalyticModelCoevolution2008,
  guoDwarfSpheroidalsCD2011, luAlgorithmsRadiativeCooling2011} for
which galaxy formation physics, although simplified, is still at the
origin of the mock galaxy properties. Finally, the simplest and most
flexible approach is based on a purely empirical modelling of galaxy
properties, sometimes called Halo Occupation Density (HOD hereafter)
\citep[e.g.][]{HOD-Seljak, HOD-Berlind,
  peacockHaloOccupationNumbers2000,
  bensonNatureGalaxyBias2000,wechslerGalaxyFormationConstraints2001,scoccimarroHowManyGalaxies2001,bullockGalaxyHaloOccupation2002,
  CLF,
  yangConstrainingGalaxyFormation2003,yangEvolutionGalaxyDarkMatter2012,valeLinkingHaloMass2004,vandenboschLinkingEarlyLatetype2003,CSMF,
  SHAM-Kravtsov, SHAM-Vale-Ostriker,
  conroyModelingLuminositydependentGalaxy2006a, Behroozi, Moster,
  guoHowGalaxiesPopulate2010, yamamotoTestingSubhaloAbundance2015,
  nuzaClusteringGalaxiesSDSSIII2013}.  The last two techniques (SAM
and HOD) both require the complete formation history of dark matter
haloes, and possibly their sub-haloes. This formation history is
described by halo `\emph{merger trees}'
\citep{roukemaSpectralEvolutionMerging1993,
  roukemaFailureSimpleMerging1993, laceyMergerRatesHierarchical1993}.
Accurate merger trees are essential to obtain realistic mock galaxy
catalogues, and constitute the backbone of SAM and HOD models.


The advantage of using SAM and HOD techniques to generate mock galaxy
catalogues is that one does not need to model explicitly the gas
component, but only the dark matter component.  The corresponding
N-body simulations are commonly referred to as `\emph{dark matter
only}' (DMO) simulations.  With growing processing power, improved
algorithms and the use of parallel computing tools and architectures,
larger and better resolved DMO simulations are becoming possible.  The
current state-of-the-art is the Flagship simulation performed for the
preparation of the Euclid mission \citep{PKDGRAV} and featured 2
trillion dark matter particles.  Such extreme simulations make
post-processing analysis tool such as merger tree algorithms
increasingly difficult to develop and to use, mostly because of the
sheer size of the data to store on disk and to load up later from the
same disk back into the processing unit memory.  In some extreme
cases, the amount of data that needs to be stored to perform a merger
tree analysis in post-processing is simply too large.  Storing just
particle positions and velocities in single precision for trillions of
particles requires dozens of terabytes per snapshot.  Another issue is
that most modern astrophysical simulations are executed on large
supercomputers which offer large distributed memory.  Post-processing
the data they produce may also require just as much memory, so that
the analysis will also have to be executed on the distributed memory
infrastructures as well.  The reading and writing of such vast amount
of data to a permanent storage remains a considerable bottleneck,
particularly so if the data needs to be read and written multiple
times.  One way to reduce the computational cost is to include
analysis tools like halo-finding and the generation of merger trees in
the simulations and run them ``\textit{on the fly}'', i.e. run them
during the simulation, while the necessary data is already in memory.

The main motivation for this work is precisely the necessity for such
a merger tree tool for future ``beyond trillion particle''
simulations.  To this end, a new algorithm that we named
\texttt{ACACIA} was designed to work on the fly within the parallel
AMR code \ramses.  One novel aspect of this work is the use of the
halo finder \phew\ \citep{PHEW} for the parent halo catalogue.
Different halo finders have been shown to have a strong impact on the
quality of the resulting merger trees \citep{SUSSING_HALOFINDER}.
\phew\ falls into the category of ``watershed'' algorithms that are
not so common in the cosmological halo finding literature.  This type
of algorithm assigns particles (or grid cells) to density peaks above
a prescribed density threshold and according to the so-called
``watershed segmentation'' of the negative density field.

This paper is structured as follows.  In Section \ref{chap:phew}, a
brief description of the \phew\ halo finder and its new particle
unbinding method is given.  Section \ref{chap:my_code} describes the
merger tree algorithm \texttt{ACACIA} and shows test results to
determine what parameters give the best results.  Using the halo
catalogue and its corresponding merger tree generated on the fly by a
cosmological N-body simulation, we use the stellar-mass-to-halo-mass
(SMHM) relation from \cite{Behroozi} to produce a mock galaxy
catalogue in the light cone. We analyse in
Section~\ref{chap:mock_catalogues} the properties of our mock galaxy
catalogue and show that the introduction of orphan galaxies improve
the comparison to observations considerably. A detailed description of
the \texttt{ACACIA} algorithm is given in Appendix
\ref{app:detailed_mergertree}, and a detailed comparison with the
other halo finding and tree-building algorithms presented in
\citet{SUSSING_HALOFINDER} is given in Appendix
\ref{app:performance_comparison}.

    \input{./tex/from-halofinding-to-mocks}
    %====================================================================
\section{Making and Testing Merger Trees}\label{chap:my_code}
%====================================================================

%============================================
\subsection{Making Merger Trees}
%============================================


The first step for any merger tree code is to identify plausible progenitor candidates for descendant clumps as well as descendant candidates for progenitor clumps.
In this algorithm, this is done by tracking up to a maximal number, labelled $n_{mb}$, of particles per progenitor clump.
The minimal number of tracker particles is given by the mass threshold for haloes.
The tracker particles of any halo are chosen to be the $n_{mb}$ particles with the lowest energy $E$:
\begin{align*}
	E = m \mathbf{v} ^2 + \phi(\mathbf{r})
\end{align*}
where $m$ is the particle's mass, $\mathbf{v}$ is the particle's velocity relative to the halo's bulk velocity, $\phi$ is the gravitational potential of the halo and $\mathbf{r}$ is the position of the particle.
If $E < 0$, a particle is considered to be energetically bound to the halo. 
Selecting the $n_{mb}$ particles with lowest energy $E$ thus corresponds to choosing the ``$n_{mb}$ \textbf{m}ost (tightly) \textbf{b}ound particles of the halo''.
This choice is made because the most strongly bound particles are expected to more likely remain within the clump between two snapshots.


For every clump in the current snapshot, the $n_{mb}$ tracker particles are found and written to file.
In the following output step, those files will be read in and sorted out: 
The clumps of the previous snapshot will be the progenitors of this snapshot. 
Based on in which descendant clump each progenitor's particles ended up in, progenitors and descendants are linked, i.e. possible candidates are identified this way.
 
Next, the main progenitor of each descendant and the main descendant of each progenitor need to be found. 
This search is performed iteratively.
A main progenitor-descendant pair is established when the main progenitor of a descendant is the main descendant of said progenitor. 
At every iteration, all descendant candidates of all progenitors that haven't found their match yet are checked;
The descendants without a matching progenitor however only move on to the next best progenitor candidate. 
For both descendants and progenitors, all candidates are ranked from ``best'' to ``worst'' based on the merit function \eqref{eq:merit}.
The iteration is repeated until every descendant has checked all of its progenitor candidates or found its match. 
Progenitors that haven't found a main descendant that isn't taken yet will be considered to have merged into their best fitting descendant candidate.

After the iteration, any progenitor that is considered as merged into its descendant will be recorded as a ``past merged progenitor''.
Only one, the most strongly bound, particle and the time of merging will be stored for past merged progenitors.
This particle is referred to as the ``galaxy particle'' of the merged progenitor.
Storing this data will allow to check in later, non-consecutive snapshots whether the progenitor has truly merged into its main descendant and to track orphan galaxies.

Then descendants that still haven't got a progenitor at this point will try to find one in non-consecutive past snapshots:
The particles that the descendant consists of are checked for being a galaxy particle of a past merged progenitor. 
The most strongly bound galaxy particle will be considered the main progenitor of the descendant under consideration.


Descendants that still haven't found a progenitor at this point are deemed to be newly formed.
This concludes the tree-making and the results are written to file.

Because every processing unit in the current implementation reads in all progenitor data, which unlike the current clump data doesn't change any more, no intricate and flexible communication structures like a peak communicator for the clump finder are necessary.
Simple collective MPI communications suffice.

Lastly, there is an option to remove past merged progenitors from the list once they merged into their main descendants too many snapshots ago.
By default, the algorithm will store them until the end of the simulation.
For the interested reader, a detailed description of the merger tree algorithm is given in appendix \ref{app:detailed_mergertree}.






    %\begin{figure}[htb]
%	\centering
%	\fbox{\includegraphics[width = .8\textwidth ]{images/cosmo/cos-part2map-npart.png}}%
%	\caption{
%		The particle distribution of the \cosmo\ dataset.
%		It is a cosmological simulation of $128^3$ dark matter particles at redshift $z=0$ with $H_0 = 70.4$ and density parameters $\Omega_m = 0.272$ and $\Omega_\Lambda=0.728$.
%	}%
%	\label{fig:cosmo_origin}
%\end{figure}




%\begin{figure}[htbp!]
%	\centering
%	\minipage[t]{0.497\textwidth}
%	\centering
%	\fbox{\includegraphics[height=\textwidth]{images/dice-two/dice-two-original-plot.png}}%
%	\caption{
%		The initial particle distribution of the \dt\ dataset. A smaller halo (subhalo 1) made of 40'000 particles is nested within a bigger halo (halo-namegiver), which contains 200'000 particles.
%	}%
%	\label{fig:dice_two_origin}
%	\endminipage%\hspace{.1cm}
%	\hspace*{\fill}
%	%
%	\minipage[t]{0.497\textwidth}
%	\centering
%	\fbox{\includegraphics[height=\textwidth, keepaspectratio]{images/dice-sub/dice-sub-original-plot.png}}%
%	\caption{
%		The initial particle distribution of the \ds\ dataset. 3 small halos (subsubhalo 1-3), made of 20'000 particles, are around a bigger halo (subhalo 1), which contains 200'000 particles. These four structures are then enveloped by an even bigger halo (halo-namegiver), made of 1'000'000 particles.
%	}%
%	\label{fig:dice_sub_origin}
%	\endminipage\hspace*{\fill} 
%\end{figure}


%============================================================================
\subsection{Testing Parameters of the Merger Tree Algorithm}\label{chap:tests}
%============================================================================

%============================================================================
\subsubsection{Methods}
%============================================================================
The current implementation of the merger tree algorithm allows for multiple free parameters for the user to choose from, which will be introduced and tested further below in this section.
Testing these parameters is not a straightforward matter, mainly because there is no ``correct solution'' which would enable a comparison and error quantification.
Nevertheless, one could define a set of quantities one deems a priori favourable for a merger tree and cross-compare these quantities obtained for varying parameters on an identical cosmological simulation.
This method was also used in the ``Sussing Merger Trees Comparison Project'' (\cite{SUSSING_COMPARISON}, \cite{SUSSING_CONVERGENCE}, \cite{SUSSING_HALOFINDER}).
Some or similar quantifications from the ``Sussing Merger Tree'' paper series are adapted in this work, as they seem sensible and even allow for a rough cross-comparison with other merger tree codes.

The following properties will be used to quantify the merger trees:

\begin{itemize}
	\item \textbf{Length of the Main Branch}
	
		The length of the main branch of $z=0$ haloes gives a most basic measure of how far back in time the formation history of these haloes can be tracked.
		Naturally, longer main branches should be considered a favourable feature for a merger tree code.
		
		In this work, the length of the main branch is defined as the number of snapshots a halo and its progenitors appear in.
		A newly formed halo at the $z=0$ snapshot, which doesn't have any progenitors, will by definition have the main branch length of $1$.
		If a halo appears to merge into another, but re-emerges at a later snapshot and is identified to do so, then the snapshots where it is missing from the halo catalogue will still be counted towards the length of the main branch as if it weren't missing. \\

	
	\item \textbf{Branching Ratio}
	
		A further simple tree quantity is the number of branches of the tree.
		The main branch is included in this count, thus the minimal number of branches for each clump at $z=0$ will be $1$.
	
%		Even though the evaluation will be conducted on the identical simulation, some parameters will effect the definitions of haloes an therefore the resulting halo catalogues.
%		As the evaluation will be conducted on the same simulation, I expect lower branching ratios to be accompanied by longer main branches and vice versa.
		As long as the halo catalogue remains unchanged, lower branching ratios mean less merging events and therefore should be accompanied by longer main branches. 
		
		In the picture of bottom-up structure formation, where larger object form through repeated mergers of smaller ones, one would expect more massive clumps to have longer main branches and a higher branching ratio. \\
		
		
	
	\item \textbf{Misidentifications, Quantified by Displacements}
	
		Since no unique correct solution exists, the same displacement statistic $\Delta_r$ as is done in \cite{SUSSING_CONVERGENCE} to quantify misidentifications is used:
%	
		\begin{align}
			\Delta_r = \frac{
				| \mathbf{r}_{k+1} - \mathbf{r}_k - 0.5 (\mathbf{v}_{k+1} + \mathbf{v}_k) (t_{k+1} - t_k) |}
				{0.5(R_{200,k} + R_{200,{k+1}} + | \mathbf{v}_{k+1} + \mathbf{v}_k | (t_{k+1} - t_k)} \label{eq:displacements}
		\end{align}
%
		where $\mathbf{r}_{k+1}$, $\mathbf{v}_{k+1}$ and $\mathbf{r}_k$, $\mathbf{v}_k$ are the position and velocity of a clump at snapshot $k+1$ and its progenitor at snapshot $k$, respectively; $t_{k+1}$ and $t_k$ are the cosmic times at which the two clumps were defined, and $R_{200}$ is the radius that encloses an overdensity of 200 times the critical density $\rho_{c} = \frac{3 H^2}{8 \pi G}$.
		
		It can be interpreted as the difference between the actual displacement $(\mathbf{r}_{k+1} - \mathbf{r}_k)$ and the expected one $(0.5 (\mathbf{v}_{k+1} + \mathbf{v}_k) (t_{k+1} - t_k) )$, normalized by an estimate of how large the displacement is allowed to be to rule out a clear misidentification.
		This estimate is the sum of the average halo size, $0.5(R_{200,k}+R_{200,k+1})$, allowing the exact clump position to vary within its own boundaries, and an estimate for the distance travelled based on the time interval and average velocities, $0.5| \mathbf{v}_{k+1} + \mathbf{v}_k | (t_{k+1} - t_k)$.
		
		Values of $\Delta_r > 1$ would indicate a misidentification, so the parameters minimising $\Delta_r$ should be preferred, provided the acceleration is approximately uniform.
		This is often not the case for subhaloes, which will shift the distribution of $\Delta_r$ towards higher values. 
		Nevertheless, it is used for subhaloes as well, since the goal is a simple cross-comparison of varying parameters, hoping to get some indication towards which parameters produce the most reliable merger trees. \\
		
		
		
		
	\item \textbf{Logarithmic Mass Growth}
	
		The logarithmic mass growth rate of haloes is approximated discretely by
		%
		\begin{align}
			\frac{\de \log M}{\de \log t} \approx \frac{(t_{k+1}+t_{k})(M_{k+1} -M_{k})}{(t_{k+1} - t_k)(M_{k+1} + M_{k})} \equiv \alpha_M(k, k+1)
		\end{align}
		%
		where $k$ and $k+1$ are a clump and its descendant, with masses $M_k$ and $M_{k+1}$ at times $t_k$ and $t_{k+1}$, respectively.
		
		To reduce the range of possible values to the interval $(-1, 1)$, \cite{SUSSING_CONVERGENCE} define
		%
		\begin{align}
			\beta_M = \frac{2}{\pi}\arctan(\alpha_M) \label{eq:massgrowth}
		\end{align}
		%
		Within the hierarchical structure formation scenario, one would expect haloes to grow over time, thus a distribution of $\beta_M$ should be skewed towards $\beta_M > 0$.
        $\beta_M \rightarrow \pm 1$ imply $\alpha_M \rightarrow \pm \infty$, indicating extreme mass growth or losses.\\
		
	\item \textbf{Mass Growth Fluctuations}
		
		Mass growth fluctuations can be quantified by using
		%
		\begin{align}
			\xi_M = \frac{\beta_M(k, k+1) - \beta_M(k-1, k)}{2} \label{eq:massfluct}
		\end{align}
		%
		where $k-1$, $k$, $k+1$ represent consecutive snapshots.
		When far from zero, it implies an extreme growth behaviour. 
        For $\xi_M\rightarrow \pm 1$, $\beta_M(k, k+1) \rightarrow \pm 1$ and $\beta_M(k-1, k) \rightarrow \mp 1$, indicating extreme mass loss followed by extreme mass growth for the upper sign, and the opposite behaviour for the lower sign.
		Within the hierarchical structure formation scenario this behaviour shouldn't occur and such an occurrence might indicate either a misidentification by the tree code or an error in the mass assignment of the halo finder.
\end{itemize}



%In order to produce evaluations that will be comparable to \cite{SUSSING_CONVERGENCE}, only clumps with mass above $5 \cdot 10^{11} \msol$ will be considered for the displacement, mass growth and mass growth fluctuation statistics.
%In all cases that will be considered, more than 3000 clumps have masses above this mass threshold.




In the evaluation, no distinction between main haloes and subhaloes is made.
Distinguishing between those two cases gives no information on which parameters are preferable that can't already be seen when no distinction is made, so for clarity's sake, the evaluations for main haloes and subhaloes individually are omitted from the main body of this work, but the figures for the displacement statistics, logarithmic mass growth and mass growth fluctuations for haloes and subhaloes separately can be found in appendix \ref{app:halo-subhalo-mtree-evals}.
















%============================================================================
\subsubsection{Parameters Influencing the Halo Catalogue}
%============================================================================

In the current implementation, there are two parameters which influence the halo catalogue aside from mass and density thresholds.
The first one concerns the mass definition of a subhalo.
By construction, the mass of a halo contains all its substructure's mass.
This isn't necessarily the case for subhaloes though.
In a hierarchical structure formation scenario, substructures are expected to contain substructures on their own.
Whether to recursively include substructure mass to their respective parent structure is a matter of choice and application. 
The influence of this choice on the merger trees is shown in figures \ref{fig:saddle_nosaddle_masses}, \ref{fig:saddle_nosaddle_mbl_nbranch} and \ref{fig:saddle_nosaddle_displacement}.
When subhaloes' masses are defined to include their respective substructure masses, the results will be labelled as \inc, or \exc\ otherwise.




A second matter of definition is in which case a particle is to be considered as bound to a clump.
The concept of ''exclusively bound`` particles, which aren't allowed to leave the spatial boundaries of their host clump, was introduced in section \ref{chap:unbinding}.
It is to be expected that demanding particles to be exclusively bound will find more unbound particles than not doing so, thus changing the subhalo catalogue.

The influence of this choice on the merger trees is also shown in figures \ref{fig:saddle_nosaddle_masses}, \ref{fig:saddle_nosaddle_mbl_nbranch}, and \ref{fig:saddle_nosaddle_displacement}, along with the influence of the previously described \inc\ and \exc\ mass definitions.
When bound particles are allowed to leave the clump's boundaries, the results will be labelled as \nosad, or \sad\ otherwise.







%============================================================================
\subsubsection{Dataset Used for Testing}
%============================================================================
All tests are performed on the same dark matter only simulation which contains $256^3 \approx 1.7\cdot 10^7$ particles of identical mass $m_p = 1.55\cdot 10^9\msol$. 
The Hubble constant $H_0 = 70.4$ km s$^{-1}$Mpc$^{-1}$ and density parameters $\Omega_m = 0.272$ and $\Omega_\Lambda = 0.728$ were used. 
The density threshold for clump finding was chosen to be 80$\rho_c$  and the saddle threshold for halos was set to 200$\rho_c$, where $\rho_c = \frac{3 H^2}{8 \pi G}$ is the cosmological critical density. 
Only clumps with at least 10 particles were kept.

The output strategy was chosen as follows:
As virtually no haloes were found before $a\leq 0.1$, only few snapshots were stored up to $a=0.1$ in steps of $\Delta a \approx 0.02$.
From this point on, snapshots were created every $\Delta t \approx 0.3$ Gyrs up until $a = \tfrac{1}{3}$, after which a smaller time interval of $\Delta_t \approx 0.2$ Gyrs were chosen.
This choice resulted in 67 snapshots to get to $z = 0$.
The simulation was then continued for 3 further snapshots with $\Delta t \approx 0.2$ Gyrs to ensure that the merging events at $z = 0$ are actually mergers and not clumps that will re-emerge later.

A visualisation of the merger tree of the most massive main halo of this simulation is shown in figure \ref{fig:mergertree}.
Stunningly, even for a relatively low resolution simulation like the one used, incredibly complex formation histories can be uncovered and followed back to the first snapshot with identifiable haloes.
Note that this is only the tree of the central halo, not containing any subhaloes that are still identifiable as such.


\begin{figure}[H]
    \begin{minipage}[c]{0.6\textwidth}
    	\includegraphics[width=25cm, keepaspectratio,angle=90,origin=c]{images/mergertree.pdf}%
    \end{minipage}\hfill
    \begin{minipage}[c]{0.4\textwidth}
    	\caption{
    		The merger tree of the most massive central halo in the simulation, obtained with the parameters \exc\ and $n_{mb}=1000$.
    		The redshift at the time of the snapshot is given on the $x$-axis, the $y$-axis has no physical meaning.
    	}%
    	\label{fig:mergertree}
    \end{minipage}
\end{figure}









\subsubsection{Influence of the Definition of Subhalo Mass}


\begin{figure}[H]
	\centering
	\includegraphics[width=\textwidth, keepaspectratio]{images/saddle-vs-nosaddle/mass_growth_and_fluctuation.pdf}%
	\caption{
		Logarithmic mass growth and mass growth fluctuation distributions for progenitor - descendant pairs or three consecutive nodes in a branch, respectively, with masses above $5\cdot 10^{11}\msol$ throughout the entire simulation for the halo catalogue influencing parameters: 
		whether subhalo particles are included (\inc) or excluded (\exc) in the clump mass of satellite haloes, and whether to consider particles which might wander off into another clump as bound (\nosad) or not (\sad).
		The distribution is computed as a histogram which is normalised by the total number of events found.
	}%
	\label{fig:saddle_nosaddle_masses}
\end{figure}

\begin{figure}[p]
	\centering
	\minipage[t]{0.49\textwidth}
	\centering
	\includegraphics[width=\textwidth]{images/saddle-vs-nosaddle/length_of_main_branch.pdf}
	\endminipage%\hspace{.1cm}
	\hspace*{\fill}
	%
	\minipage[t]{0.49\textwidth}
	\centering
	\includegraphics[width=\textwidth, keepaspectratio]{images/saddle-vs-nosaddle/number_of_branches.pdf}%
	\endminipage\hspace*{\fill} 
	\caption{
		Length of main branches, defined as the number of snapshots this clump appears in, and the number of branches including the main branch, for the halo catalogue influencing parameters of $z=0$ clumps: 
		whether subhalo particles are included (\inc) or excluded (\exc) in the clump mass of satellite haloes, and whether to consider particles which might wander off into another clump as bound (\nosad) or not (\sad).
		Four distributions are shown, for four different ranges of numbers of particles at $z=0$ exclusively assigned to the clump: less then 100 (top), 100-500, 500-1000 and more than 1000 (bottom), where each particle has mass $m_p = 1.55 \cdot 10^9 \msol$.
		The distribution is computed as a histogram which is normalised by the total number of events found per particle count group.
	}%
	\label{fig:saddle_nosaddle_mbl_nbranch}
\end{figure}

\begin{figure}[H]
	\centering
	\minipage[t]{0.49\textwidth}
		\includegraphics[width=\textwidth]{images/saddle-vs-nosaddle/displacements.pdf}%
		\caption{
			Distribution of the displacements for progenitor - descendant pairs with masses above $5\cdot 10^{11}\msol$ throughout the entire simulation for the halo catalogue influencing parameters: 
			whether subhalo particles are included (\inc) or excluded (\exc) in the clump mass of satellite haloes, and whether to consider particles which might wander off into another clump as bound (\nosad) or not (\sad).
			The distribution is computed as a histogram which is normalised by the total number of events found.
		}%
		\label{fig:saddle_nosaddle_displacement}
	\endminipage%\hspace{.1cm}
	\hspace*{\fill}
	%
	\minipage[t]{0.49\textwidth}
		\centering
		\includegraphics[width=\textwidth]{images/ntracers/displacements.pdf}%
		\caption{
			Distribution of the displacements for progenitor - descendant pairs with masses above $5\cdot 10^{11}\msol$ throughout the entire simulation for varying numbers of clump tracer particles $n_{mb}$.
			The distribution is computed as a histogram which is normalised by the total number of events found.
		}%
		\label{fig:ntracers_displacement}
	\endminipage
\end{figure}













%===============================================================
%\subsubsection*{Length of Main Branch and Branching Ratio}
%===============================================================

\begin{table}[!ht]
	\begin{center}

		{\small 
			\begin{tabular}[c]{l | p{1.8cm} | p{1.8cm} | p{1.8cm} | p{1.8cm} |}
											&	\inc\ \sad & \exc\ \sad\ & \inc\ \nosad\ & \exc\ \nosad\ \\ 
				
				\hline
				%
				total clumps	 						&	16262	& 	16262	& 	17242	&	17242 	\\
				
				%		
				max number of particles in a clump 		&	414570	& 	414570	& 	271438	&	271438 	\\
				
				%		
				median number of particles in a clump 	&	84		& 	83		& 	93		&	93 		\\
				
				%
				\hline
				%		
				average main branch length group I		&	22.980	& 	23.153	& 	20.426	&	20.527 	\\
				
				%
				average main branch length group II		&	48.653	& 	48.835	& 	48.132	&	48.642 	\\
				
				%		
				average main branch length group III	&	54.358	& 	54.715	& 	54.305	&	54.904 	\\
				
				%
				average main branch length group IV		&	54.546	& 	53.958	& 	56.110	&	56.265 	\\
				
				%
				\hline
				%		
				average number of branches group I		&	1.318	& 	1.327	& 	1.230	&	1.230 	\\
				
				%
				average number of branches group II		&	3.480	& 	3.448	& 	3.632	&	3.603 	\\
				
				%		
				average number of branches group III	&	8.466	& 	8.670	& 	8.673	&	8.661 	\\
				
				%
				average number of branches group IV		&	29.771	& 	29.457	& 	28.783	&	28.607 	\\	
				
				%
				\hline	
			\end{tabular}
		}
		\caption{
			Average data for all clumps at $z=0$ for the four halo catalogue modifying parameter pairs: whether subhalo particles are included (\inc) or excluded (\exc) in the clump mass of satellite haloes, and whether to consider particles which might wander off into another clump as bound (\nosad) or not (\sad).
			The groups I, II, III and IV are defined as clumps that contain less then 100, 100-500, 500-1000 or more than 1000 particles, respectively.
		}
		\label{tab:saddle_nosaddle}
	\end{center}	
\end{table}

In accordance to the hierarchic structure formation picture, more massive clumps tend to have longer main branches and a higher branching ratio in all cases.
This is clearly visible from the average length of the main branch and the average branching ratio of clumps at $z=0$, binned in four groups by their mass, which are given in table \ref{tab:saddle_nosaddle}.
The average main branch length for clumps with more that 500 particles is $\sim 55$, meaning that on average, haloes with mass above $\sim 7.75 \cdot 10^{11} \msol$ can be traced back to redshift $\sim 3$.

The length of the main branches for the same four mass bins of clumps are shown in figure \ref{fig:saddle_nosaddle_mbl_nbranch}.
Interestingly, small clumps with less than 100 particles seem to have a somewhat constant formation rate from $z \sim 2$ until $z \sim 0.2$.
Furthermore they too can be traced back to high redshifts, indicating good performance of the merger tree algorithm.

Whether subhalo masses are computed in an \exc\ or \inc\ manner seems to have negligible effect on the branching ratio and the length of main branch.

The \sad\ definition of bound particles however tends to result in longer main branch lengths for small clumps with less than 100 particles.
This might be explained by the fact that in general, when \sad\ is applied, subhaloes which are at the bottom of the clump hierarchy will tend to contain less bound particles compared to when \nosad\ is used, and have shorter lifetimes because they are found to have merged into their hosts earlier.
Therefore clumps with more than 100 particles with the \nosad\ condition might be moved to the lower mass bin of $\leq$ 100 particles when \sad\ is used, thus increasing the fraction of clumps with high main branch lengths (lengths of 50-60), as well as the number of branches.
Evidence of the earlier merging can be seen in the overall higher number of branches in the right column of figure \ref{fig:saddle_nosaddle_mbl_nbranch}, as well as the average values, total clump numbers and the median particle numbers in a clump given in table \ref{tab:saddle_nosaddle}.
%The longer main branch length can be understood by considering that with a clump definition which in general has less particles, some clumps which would otherwise be sorted in a higher mass bin are sorted in a lower one instead.
%This should increase both the length of main branches and the number of branches for lower mass clumps (below 100 particles) and decrease them for high mass clumps (above 1000 particles), as table \ref{tab:saddle_nosaddle} and figure \ref{fig:saddle_nosaddle_mbl_nbranch}.
%For intermediate masses, meaning 100-1000 particles in a clump, the distributions and average values don't vary significantly.








%==========================================
%\subsubsection*{Misidentifications}
%==========================================

The displacement statistic used to quantify misidentifications in figure \ref{fig:saddle_nosaddle_displacement} indicates that no parameter choice results in a obviously ``wronger'' progenitor-descendant pairing.
Even though there are some  $\Delta_r > 1$, indicating that there might be misidentifications present, it can't be determined easily whether they truly are.
When the displacement statistic is calculated for haloes and subhaloes separately, which is shown in appendix \ref{app:halo-subhalo-mtree-evals}, no halo descendant-progenitor pair shows a displacement above 1, indicating no obvious misidentifications, yet a significant fraction of subhalo progenitor-descendant pairs have a high displacement value $\Delta_r > 1$.
However, the displacement statistic is calculated assuming a uniform acceleration, which is not valid for subhaloes.
On the other hand, the fact that any parameter pair used found at least some clumps with more than 1000 particles at $z=0$ with main branch length of unity, which is the case when it doesn't have any progenitor, and thus essentially ``appearing out of nowhere'', is strongly suggesting present misidentifications.




%=========================================================================
%\subsubsection*{Logarithmic Mass Growth and Mass Growth Fluctuations}
%=========================================================================

The logarithmic mass growth in figure \ref{fig:saddle_nosaddle_masses} shows that as expected, the distribution is indeed skewed towards $\beta_M > 0$. 
When the \inc\ parameter is used, the distribution of mass growth contains a few more extreme mass growths and losses $(\beta_M \rightarrow \pm 1)$, as well as some high mass growth fluctuations $(\xi_M \rightarrow \pm 1)$ (see figure \ref{fig:saddle_nosaddle_masses}).
When the \nosad\ parameter is used, noticeably more extreme mass growth $(\beta_M \rightarrow \pm 1)$ and mass growth fluctuations $(\xi_M \rightarrow \pm 1)$ occur.






%==========================================
%\subsubsection*{Conclusion}
%==========================================
In conclusion, whether to use \inc\ or \exc\ mass definitions for subhaloes shows very little effect on the merger trees.
Based on the fewest extreme mass growth fluctuations, the \sad\ parameter is clearly preferable. 
%With \sad\ in use, the \exc\ parameter leads to less clumps over 1000 particles with main branch length of 1, and a little fewer extreme mass growths and fluctuations, which is why I conclude that the combination of \exc\ and \sad\ parameters should create the most reliable merger trees.





	













%===================================================================
\subsection{Influence of the Number of Tracer Particles Used}
%===================================================================

\begin{table}[ht]
	\begin{center}
		{\small 
		\begin{tabular}[c]{l | p{1cm} | p{1cm} | p{1cm} | p{1cm} | p{1cm} | p{1cm} | p{1cm} |}
			$n_{mb}=$								&	1 		& 	10 		& 	50 		& 	100 	& 200 	& 500 		& 1000 \\
			\hline
	%
%			total clumps at $z=0$					&	16262	& 	16262	& 	16262	&	16262 	& 16262  & 16262 	& 16262 \\
	%		
%			max NoP in a clump 		&	414570	& 	414570	& 	414570	&	414570 	& 414570 & 414570 	& 414570  \\ 	
	%		
%			median NoP in a clump 	&	83		& 	83		& 	83		&	83  	& 83	 & 83 	& 	83  \\	
	%
%			\hline
	%		
			average MBL group I		&	24.188	& 	24.330	& 	23.567	&	23.353 	& 23.153 & 22.876 	& 22.656  \\	
	%
			average MBL group II		&	50.399	& 	50.116	& 	49.472	&	49.122 	& 48.835 & 48.777 	& 48.762  \\	
	%		
			average MBL group III	&	55.233	& 	54.863	& 	53.264	&	54.059 	& 54.715 & 54.327 	& 54.165  \\	
	%
			average MBL group IV		&	56.690	& 	54.884 	& 	52.345	&	52.900 	& 53.958 & 55.761 	& 56.448  \\
	%
			\hline
	%		
			average NoB group I		&	1.228	& 	1.305	& 	1.296	&	1.305 	& 1.327  & 1.357 	& 1.367  \\	
	%
			average NoB group II		&	2.699	& 	3.062	& 	3.265	&	3.337 	& 3.448  & 3.586 	& 3.596  \\	
	%		
			average NoB group III	&	6.625	& 	7.229	& 	8.051	&	8.206 	& 8.670  & 8.914 	& 9.121  \\	
	%
			average NoB group IV		&	20.407	& 	25.237	& 	27.288	&	28.554 	& 29.457 & 30.443 	& 31.420  \\	
	%
			\hline	
		\end{tabular}
		}
	\caption{
		Average data for all clumps at $z=0$ for varying numbers of clump tracer particles $n_{mb}$. 
		The groups I, II, III and IV are defined as clumps that contain less then 100, 100-500, 500-1000 or more than 1000 particles, respectively.
		``MBL'' is an abbreviation for ``main branch length'', ``NoB'' stands for ``number of branches''.
        % ``NoP'' for ``number of particles''.
		}
	\label{tab:ntracers}
	\end{center}	
\end{table}
\begin{table*}
  
  \caption{Number of dead trees pruned from the merger tree catalogue
    for varying numbers of tracer particles $n_{\rm mb}$ throughout
    all snapshots.  ``LIDIT'' is an abbreviation for ``last
    identifiable descendant in tree''.  For a LIDIT, no descendant
    could have been identified throughout the simulation and
    consequently the corresponding tree is considered dead and pruned
    from the merger tree catalogue. LIDITS are obviously a spurious
    feature of the merger tree algorithm.
    \label{tab:ntracers-pruning}
  }
  
  %	\begin{center}
  {\small 
    \begin{tabular}[c]{l | p{1cm} | p{1cm} | p{1cm} | p{1cm} | p{1cm} | p{1cm} | p{1cm} |}
      $n_{\rm mb}=$				&	1 	& 	10 	& 	50 	& 	100 	& 200 	& 500 	& 1000  \\
      \hline	
      dead trees pruned from tree catalogue	&	33924	&	23091	&	22146	& 	22131 	& 22130 & 22129 & 22129 \\	
      %
      highest particle number of a LIDIT	&	1369	&	236	&	236	&	236 	& 236 	& 157 	& 157  	\\	
      %
      median particle number of a LIDIT		&	19	&	20	&	20	&	20 	& 20 	& 20 	& 20  	\\
      %
      LIDITs with >100 particles pruned 	&	513	&	42	&	32	&	26 	& 25 	& 24 	& 24  	\\
      %
%      total number of jumpers                 &  20176    &   20905   &   22074   &   22041   & 20970 & 19307 & 18249 \\
      \hline
    \end{tabular}
  }
  %	\end{center}	
\end{table*}


\begin{figure}[H]
	\centering
	\includegraphics[width=\textwidth, keepaspectratio]{images/ntracers/mass_growth_and_fluctuation.pdf}%
	\caption{
		Logarithmic mass growth and mass growth fluctuation distributions for progenitor - descendant pairs or three consecutive nodes in a branch, respectively, with masses above $5\cdot 10^{11}\msol$ throughout the entire simulation for varying numbers of clump tracer particles $n_{mb}$.
		The distribution is computed as a histogram which is normalised by the total number of events found.
	}%
	\label{fig:ntracers_masses}
\end{figure}

\begin{figure}[p]
	\centering
	\minipage[t]{0.49\textwidth}
	\centering
	\includegraphics[width=\textwidth]{images/ntracers/length_of_main_branch.pdf}
	\endminipage%\hspace{.1cm}
	\hspace*{\fill}
	%
	\minipage[t]{0.49\textwidth}
	\centering
	\includegraphics[width=\textwidth, keepaspectratio]{images/ntracers/number_of_branches.pdf}%
	\endminipage%\hspace*{\fill}
	\caption{
		Length of main branches, defined as the number of snapshots this clump appears in, and the number of branches including the main branch of $z=0$ clumps for varying numbers of clump tracer particles $n_{mb}$.
		Four distributions are shown, for four different ranges of numbers of particles at $z=0$ exclusively assigned to the clump: less then 100 (top), 100-500, 500-1000 and more than 1000 (bottom), where each particle has mass $m_p = 1.55 \cdot 10^9 \msol$.
		The distribution is computed as a histogram which is normalised by the total number of events found per particle count group.
	}%
	\label{fig:ntracers_mbl_nbranch}
\end{figure}













%===================================================================
%\subsubsection*{Length of Main Branch and Branching Ratio}
%===================================================================

The average number of branches and average main branch lengths are shown in table \ref{tab:ntracers}.
The average number of branches increases with the number of tracers used, the case for $n_{mb}=10$ for clumps with less than 100 particles being the only exception.
The average main branch length decreases for the two lower mass clump bins (less than 500 particles).
This can also be seen in the top two rows of figure \ref{fig:ntracers_mbl_nbranch}, where the length of the main branches and the number of branches are plotted.
This indicates that more mergers were detected.
Counter-intuitively, this can be seen as a sign that more reliable trees are created with increasing $n_{mb}$.
Recall that progenitors from adjacent snapshots are given priority over non-adjacent ``jumpers''.
By tracking more particles per clump, more candidates can be expected to be found, which is supported by the fact that the number of jumpers in the simulations decreasing with increasing $n_{mb}$ (see table \ref{tab:ntracers-pruning}).
Considering that also being a main progenitor to any descendant is given priority over being merged into the main descendant of the progenitor, it should be safe to say that it should be true merging events that have been misidentified by tracking fewer $n_{mb}$ particles.
%A contributing factor might be that with more tracers, more progenitor-descendant links of adjacent snapshots are made and less connections across multiple snapshots are necessary, reducing the risk of assigning the wrong formation history across multiple snapshots.
%However, since direct progenitors from adjacent snapshots are given priority to non-adjacent ones, misidentifications might just as well be introduced by this behaviour.
%Table \ref{tab:ntracers-pruning} also shows that for $n_{mb}>100$, the total number of connections across multiple snapshots (``jumpers'') starts to decrease.

When a clump has no descendant candidates at all, its tree is removed from the list of trees.
How many of these trees have been pruned throughout the simulation is shown in table \ref{tab:ntracers-pruning}, as well as the particle number of the most massive pruned clump, the median particle number of pruned clumps and the number of clumps containing more than 100 particles that have been pruned.
With increasing $n_{mb}$, the number of pruned clumps, the highest particle number of a pruned clump, and the number of clumps with more than 100 particles decreases.
Notice that for $n_{mb}=1$, there is a drastic increase in all these three quantities.
In particular, clumps with more than 1000 particles are pruned, meaning that haloes with mass above $1.5\cdot 10^{12} \msol$ simply vanished between two snapshots.
These statistics indicate furthermore that with increasing number of tracing particles, more merging events are detected.









%===============================================
%\subsubsection*{Misidentifications}
%===============================================
The displacements in figure \ref{fig:ntracers_displacement} show virtually no differences.
The only noticeable difference is towards the high end of $\Delta_r$, where $n_{mb}=1$ and $n_{mb}=10$ have a small peak further out than the other choices for $n_{mb}$.










%====================================================================
%\subsubsection*{Logarithmic Mass Growth and Mass Growth Fluctuations}
%====================================================================
The logarithmic mass growth and mass growth fluctuations in figure \ref{fig:ntracers_masses} show that these distributions mostly overlap, but extreme growths $(\beta_M \rightarrow \pm 1)$ and fluctuations $(\xi_M \rightarrow \pm 1)$ decrease with increasing $n_{mb}$.






%===============================================
%\subsubsection*{Conclusion}
%===============================================

It seems that $n_{mb} = 100-200$ is a good compromise between computational efficiency and good results.
Note that for this simulation, the median number of particles in $z=0$ clumps was 83, meaning that with $n_{mb} = 100$, more than half of identified clumps were being tracked by every particle they contain.













\subsection{Outlook}

Based on the previously shown results, the current implementation of the merger tree algorithm seems to perform well.
The shapes of the logarithmic mass growths and mass growth fluctuations in figures \ref{fig:saddle_nosaddle_masses} and  \ref{fig:ntracers_masses} as well as the distributions of lengths of main branches and number of branches are in good agreement with the results from other merger tree codes, which have been compared in \cite{SUSSING_HALOFINDER}. 
However, there are still some unanswered conceptual questions and possible algorithm optimisations to be discussed.

On the conceptual side, when linking progenitors and descendants across multiple snapshots, one must ask:
How far in the future or in the past does one need to look for a descendant or progenitor, respectively?
At what point should one assume that the tracked progenitor is really dissolved and definitely won't reappear at later times?
The current implementation only contains the option to forget past merged progenitors after a user defined number of snapshots has passed, but by default, it will track them until the simulation ends.
By not removing orphans at all and using them to link descendants with progenitors across multiple snapshots, misassignments are enabled, leading to wrong formation histories.

Two possible solutions would be the following:

\begin{enumerate}
    \item Estimate the time a clump would require to completely merge into its parent structure, after which the progenitor shouldn't be tracked anymore. This is for example done in \cite{Moster}, where they compute the dynamical friction time $t_{df}$ of a merged subhalo based on the orbital parameters found at the last snapshot where this subhalo was identified:
    \begin{equation}
        t_{df} = \alpha_{df} \frac{V_{vir} r_{sat}^2}{G M_{sat}\ln \Lambda} \label{eq:dynamical_friction_time}
    \end{equation}
    where $r_{sat}$ is the distance between the centres of the main halo and of the subhalo, $M_{sat}$ is the mass of the subhalo, $\ln\Lambda = (1 + M_{vir}/M_{sat})$ is the Coulomb logarithm, $M_{vir}$ is the virial mass of the main halo, $V_{vir}$ is the circular velocity of the main halo at the virial radius and $\alpha_{df} = 2.34$. 
    A smaller subhalo inside a main halo experiences dynamical friction because of its gravitational attraction: 
    At any given moment, it attracts the particles of the host towards the point in space where it currently resides, but because the subhalo itself is in orbit, it will move away from that point, thus leaving a slightly denser trail along the path it moves.
    The gravitational attraction from this trail on the other hand will eventually slow it down and cause it to fall into the main halo's centre.
    
    Another possibility would be to use the fitting formula for the merger timescale of galaxies in cold dark matter models by \cite{merger_timescales}.
    
    
    \item The particle used to track a past merged progenitor is also the same particle that an orphan galaxy is assigned to.
    In principle, it should be possible to define some galaxy merging cross-sections such that the probability of a collision between an orphan galaxy and a non-orphan galaxy which will result in a galaxy merger can be computed.
    Unknown parameters of these cross-sections should be able to be calibrated using N-body simulations.
    After a collision, one could remove the orphan from future snapshots.
\end{enumerate}



From a technical viewpoint, one clear bottleneck in the current merger tree algorithm is the requirement to write progenitor particles and data to file and read them back in and sort them out at a later snapshot.
An elegant solution would be to permanently store the clump IDs of particles in memory, however this would require an extra integer per particle in the simulation, which becomes prohibitively expensive for large simulations not only because it would need a lot of memory, but also because more data needs to be communicated between MPI tasks.

An option would be to track which particles left each task's domain and which particles entered between two snapshots.
The clump IDs of particles would still be read and written to and from files, but it would minimise the sorting part of the algorithm where each MPI task figures out which tracker particles it contains.
The necessary data of particles that left or entered new domains between snapshots could then be communicated with one collective MPI communication, provided they've been tracked in a clever manner.

Another option would be to change the amount of data each MPI task needs to read in.
Currently, every MPI task reads in and writes to one shared file using MPI reading and writing routines in order to maximally make use of the parallel architecture.
Instead, each task could write its own file.
Meanwhile, between snapshots, the maximal velocity of any particle should be traced.
This way, once the simulation advances to the next snapshot, it would be possible to estimate the maximal distance any particle could've travelled.
Provided every MPI task has knowledge on how the entire computational domain is split between tasks, it could skip reading in data written by tasks where no particle currently in this task's domain could have come from.
This would however probably require a more sophisticated communication for progenitor data such as their mass or descendant candidates.
(Currently, because every MPI task reads in all the progenitor data, this communication are simple collective scatter and gather operations.)
Furthermore, the situation will get more complicated if the domain decomposition changes its shape between snapshots to e.g. load balance.





%Dynamical friction plays a crucial role in the formation and
%evolution of galaxies. During the merger of two dark matter halos,
%galaxies in a less massive halo will become the satellite galaxies of
%the more massive one. These satellite galaxies gradually lose their
%energy and angular momentum under the action of dynamical
%friction and are predestined to sink to the center of the massive
%dark matter halo if they are not disrupted by the tidal force.
%Dynamical friction takes effect through interaction of galaxies
%with background dark matter particles. Chandrasekhar (1943) gave
%a description of this phenomenon for an idealized case where a
%rigid object moves through a uniform sea of collisionless matter
%particles. This description can be applied to the case of a satellite
%galaxy moving in a dark matter halo. The orbits of dark matter
%are deflected by the galaxy, which produces an enhancement of
%dark matter density behind the galaxy. Consequently, the galaxy
%suffers a steady deceleration by the drag of the wake and will
%eventually merge to the central galaxy of the dark matter halo.
%The merger timescale, i.e., the time elapsing between entering
%the virial radius of the dark matter halo and final coalescence of
%satellite and central galaxy, can be derived using Chandrasekhar’s
%formula (see, e.g., Binney & Tremaine 1987). In addition, taking
%into account the dependence on the orbital circularity, Lacey &
%Cole (1993) derived the following expression for the merger time-
%scale of a satellite galaxy orbiting around a massive halo with
%circular velocity




    %============================================================================
\section{Testing Mock Galaxy Catalogues}\label{chap:mock_catalogues}
%============================================================================


%============================================================================
\subsection{Methods}
%============================================================================


The primary quantity a mock galaxy catalogue must reproduce are stellar mass functions $\Phi(M_*)$, which give the number density of central galaxies with stellar mass $M_*$. 
The stellar mass functions obtained using the merger tree algorithm and the SMHM relation \eqref{eq:behroozi_SMHM} are showed and discussed in section \ref{chap:smf}


The second test of the mock galaxy catalogues is whether the galaxy clustering of the Universe is reproduced.
A commonly used measure of clustering is the two-point correlation function (2PCF) $\xi(r)$, which according to the cosmological principle should be isotropic and thus a function of distance $r$ as opposed to position $\mathbf{r}$.
The two-point correlation function can be interpreted as the excess probability of finding a galaxy in a volume element at a separation $r$ from another galaxy, compared to what is expected for a uniform random distribution.
It can be computed via inverse Fourier transform of the power spectrum $P(k)$ \parencite{Mo}, which itself can be obtained from the Fourier transform of the density contrast field $\delta(\mathbf{r})$:
%
\begin{align}
	\delta_\mathbf{k} = \frac{1}{V}\int e^{i\mathbf{kr}} \delta(\mathbf{r}) \de ^3 \mathbf{r}
\end{align}
%
with
%
\begin{align}
	\delta(\mathbf{r}) = \frac{\rho(\mathbf{r})}{\langle \rho(\mathbf{r}) \rangle } - 1
\end{align}
%
Where $\rho(\mathbf{r})$ is the galaxy density field and $\langle \rho(\mathbf{r}) \rangle$ is the mean density, $V = L^3$ is the volume of a large box on which the density field is assumed periodic, and $k=\frac{2\pi}{L}(i_x, i_y, i_z)$, where $i_x, i_y, i_z$ are integers.

The power spectrum $P(k)$ and the 2PCF $\xi(r)$ are given by
%
\begin{align}
	P(k)    &= V \langle |\delta_\mathbf{k}|^2 \rangle \\
	\xi(r)  &= \frac{1}{(2\pi)^3} \int e^{-i\mathbf{kr}} P(k) \de^3 \mathbf{k} \\
		    &= \frac{1}{2\pi^2} \int\limits_0^{\infty} P(k) \frac{\sin(kr)}{kr} k^2 \de k \label{eq:corr_1d_int}
\end{align}

The simulation box is divided in a uniform grid of $1024^3$ cells and the mass is distributed using a cloud-in-cell interpolation scheme to obtain the density field.
The Fourier transforms are performed using the FFTW library \parencite{FFTW05}.
Instead of first averaging the tree-dimensional power spectrum $P(\mathbf{k})$ to obtain a one-dimensional power spectrum $P(k)$ and then integrating it following eq. \eqref{eq:corr_1d_int}, first the three-dimensional correlation function $\xi(\mathbf{r})$ is computed by computing the inverse Fourier transform on the three-dimensional power spectrum $P(\mathbf{k})$ and then $\xi(\mathbf{r})$ is averaged over all angular directions to obtain $\xi(r)$.
This has the advantage of not having to perform an integral to infinity while only a finite sample is present.


Once the real space correlation function is known, the projected correlation function $w_p(r_p)$ can be derived by integrating $\xi(r)$ along the line of sight \parencite{Moster2010}:
%
\begin{align}
	w_p(r_p) 
		= 2 \int\limits_{0}^{\infty} \de r_{||} \xi\left( \sqrt{r_{||}^2 + r_{p}^2} \right) 
		= 2 \int\limits_{r_p}^{\infty} \de r \frac{ r \xi\left( r \right) } {\sqrt{r^2 - r_{p}^2}}
\end{align}
%
where the comoving distance $r$ has been decomposed into components parallel ($r_{||}$) and perpendicular ($r_p$) along the line of sight.
The integration is truncated at half the length of the simulation box.



The obtained correlations are showed and discussed in section \ref{chap:correlation}.




%============================================================================
\subsection{Dataset Used for Testing}\label{chap:sim_galaxy}
%============================================================================

Mock galaxy catalogues from two simulations were created, each containing $512^3 \approx 1.3\cdot 10^8$ particles.
They differ in the volume they simulate: \gsmall\ covers 69 comoving Mpc, while the second simulation, \glarge, contains 100 comoving Mpc.

With different box sizes come different mass resolutions: The particle mass for \gsmall\ is $m_p = 9.59\cdot 10^7\msol$, for \glarge\ it is $m_p = 3.09\cdot 10^8\msol$.

The cosmological parameters are taken from the 2015 Plack Collaboration results \parencite{Planck2015}: 
The Hubble constant $H_0 = 67.74$ km s$^{-1}$Mpc$^{-1}$, density parameters $\Omega_m = 0.309$, $\Omega_\Lambda = 0.691$, scalar spectral index $n_s = 0.967$, and fluctuation amplitude $\sigma_8 = 0.816$ were used.
The initial conditions were created using the \texttt{MUSIC} code \parencite{MUSIC}.


As before, the density threshold for clump finding was chosen to be 80$\rho_c$  and the saddle threshold for halos was set to 200$\rho_c$, where $\rho_c = \frac{3 H^2}{8 \pi G}$ is the cosmological critical density. 
Only clumps with at least 10 particles were kept.








%============================================================================
\subsection{Stellar Mass Functions}\label{chap:smf}
%============================================================================

The obtained stellar mass functions $\Phi(M_*)$ of central galaxies for the two simulations, \gsmall\ and \glarge, compared to observed stellar mass functions are shown in figure \ref{fig:smf}.
The abbreviations used for observational data are listed in table \ref{tab:obs_smf}.
For clarity's sake, only stellar mass functions from snapshots at redshifts which are closest to the mean redshift of the observational data are plotted.
Averaging the stellar mass function over the redshift interval made very little difference compared to choosing only one closest to the mean of the interval, which is why the average stellar mass functions were omitted from the main body of this work.
As an example, the plot of both average and a single stellar mass functions for the \gsmall\ simulation are shown in appendix \ref{app:smf_variations}.




The \gsmall\ simulation gives better results at the low mass end at $z \sim 0$, and starts to deviate noticeably around $M_* \sim 10^8 \msol$.
Using a crude estimate that the SMHM ratio $M_*/M_h \sim 10^{-1} - 10^{-2}$, together with a lower mass threshold of $10 m_p \sim 10^9 \msol$ for clumps, one can see that $M_* \sim 10^8\msol$ should be the lower mass threshold for stellar mass that is accurately resolved.
Furthermore, the ``shoulder'' of the SMF around $\log_{10}M \sim 10-12$ appears flatter. This flattening seems to produce results closer to observations in most redshift intervals.
Also, at $z\sim 0$, the high mass end of the SMF is underestimated.
Because high mass central galaxies are hosted by high mas haloes, the simulation volume just might be too small to accurately represent the statistics of the high mass halo abundance.
This is supported by the fact that the \glarge\ simulation gives slightly better results at the high mass end.

%my code seems to underestimate the high mass end. 
%However, this might be a property of the used parametrisation: 
%See for example figure 2 of \verb|https://arxiv.org/pdf/1705.06347.pdf|.
The results seem quite good and within the uncertainties of the observed data up until $z \sim 1.65$, where the deviations look like they're often outside the error bars. 
It gets worse with increasing redshift.
However, seeing how in almost every case the \gsmall\ yields better results than \glarge, the high redshift SMFs should improve with increased resolution.








\input{tables/observed_smfs}



\begin{figure}[H]
	\centering
	\includegraphics[width=\textwidth]{images/smf/smf_both_sims.pdf}%
	\caption{
		Obtained stellar mass functions $\Phi(M_*)$ of central galaxies for the two simulation datasets \gsmall\ and \glarge, described in \ref{chap:sim_galaxy} with boxsize of 69 and 100 Mpc, respectively, compared to observed stellar mass function.
		The abbreviations used for observational data are listed in table \ref{tab:obs_smf}.
	}%
	\label{fig:smf}
\end{figure}


















%============================================================================
\subsection{Correlation Functions}\label{chap:correlation}
%============================================================================

The obtained 2PCF $\xi(r)$ and the projected correlation function $w_p(r_p)$ at $z\sim 0$ are shown in figure \ref{fig:correlations} for both the \gsmall\ and \glarge\ simulations, and they are compared to observational results from \cite{LiWhite} and \cite{Correlation1}.

\begin{figure}[H]
	\centering
	\includegraphics[width=\textwidth, keepaspectratio]{images/correlations.pdf}%
	\caption{
		The obtained 2PCF $\xi(r)$ and projected correlation function $w_p(r_p)$ for the \gsmall\ (dashed lines) and \glarge\ (dotted lines) simulations, both including and excluding orphan galaxies, compared to best power law fits of the 2PCF from \cite{LiWhite} and \cite{Correlation1} and the projected correlation function  from \cite{LiWhite} (solid lines).
	}%
	\label{fig:correlations}
\end{figure}



In all cases, including orphan galaxies produced correlation functions closer to observations.
\cite{crisis} also found that the inclusion of orphan galaxies for mass-based SHAM models may improve the clustering statistics of mock galaxy catalogues, particularly so at small scales.
The 2PCF obtained from the \glarge\ simulation even can reproduce the best power law fit from observations quite well for about two orders of magnitude of $r\sim 0.2 - 20$ Mpc.
Noticeably for both simulations the correlation functions start with very similar values regardless of whether orphan galaxies are included or not for small $r$. 
As the distance $r$ increases, so does the difference between the two cases, and starts decreasing around $r\sim 1-2 $ Mpc.
After $r\sim 10$ Mpc, the difference becomes very small.
This behaviour may be explained by considering that one would expect orphan galaxies to tend to be located within host halos, not somewhere in a void all by themselves, thus contributing to the correlations at small distances stronger than at large distances.


Any case underestimates the projected correlation function, but as for the 2PCF, the \glarge\ simulation with orphan galaxies included in the computation of the correlation comes closest.
Because $w_p$ is computed by numerically integrating the previously obtained $\xi(r)$, part of the reason why $w_p$ might be underestimated is the propagation of errors.
$w(r_p)$ is computed by integrating $\xi(r)$ from $r=r_p$ to $r\rightarrow \infty$, meaning that the underestimated $\xi(r)$ at large scales $r\gtrsim 20$ Mpc is included in the integration for every $r_p$.
Another reason might be that while technically the integration should be performed to infinity, it was truncated at half of the length of a simulation box.


The \glarge\ simulation gives better correlation functions, which might be due to the fact that a bigger volume was simulated.
A volume of 69 Mpc might just be too small to properly represent a homogeneous, isotropic chunk of the Universe.
















%%=========================================================
%\subsection{Outlook}
%%=========================================================
%
%Given that the mock galaxy catalogues in this work were created using simulations with relatively low spatial and mass resolution, the obtained correlation functions and stellar mass functions are satisfactory.
%A higher spatial resolution should improve the clustering statistics, and together with a higher mass resolution the stellar mass functions of central galaxies should also improve.

    %=================================================
\section{Conclusion}\label{chap:conclusion}
%=================================================


We presented \texttt{ACACIA}, a new  algorithm to identify dark matter
halo  merger trees,  which is  designed  to work  {\it on-the-fly}  on
systems  with  distributed  memory architectures,  together  with  the
adaptive mesh refinement code \ramses\ with its {\it on-the-fly} clump
finder  \phew.  Clumps  of dark  matter are  tracked across  snapshots
through a user-defined  maximum number of most bound  particles of the
clump  $n_{mb}$.   We found  that  using  $n_{mb} \simeq  200$  tracer
particles is  a safe choice  to obtain  robust results for  our merger
tree  algorithm,  while   not  being  computationally  unrealistically
expensive.  We  also recommend adopting a  conservative mass threshold
of 200  particles per  clump to get  rid of a  few rare  spurious dead
branches that would need to be pruned from the halo catalogue anyway.

Additionally,  we examined  the  influence of  various definitions  of
substructure  properties on  the  resulting merger  trees. Whether  we
define substructures to contain their respective substructures' masses
or not  had negligible effect  on the merger trees.   However defining
particles  to  be  strictly  gravitationally  bound  to  their  parent
substructure  (by requiring  that  particles can't  leave the  spatial
extent of that  substructure) leads to better results,  with much less
extreme  mass growths  and extreme  mass growth  fluctuations of  dark
matter clumps.  We recommend to  use this strictly bound definition as
the  preferred  definition for  robust  merger  trees.  The  resulting
merger  trees  are  in   agreement  with  the  bottom-up  hierarchical
structure formation picture  for dark matter haloes.  The merger trees
of massive  haloes at $z=0$ have  more branches than their  lower mass
counterparts.  Their  formation history  can often  be traced  to very
high redshifts.

Once a progenitor  clump is merged into  a descendant, \texttt{ACACIA}
keeps track of the progenitor's most strongly bound particle, called the
``orphan particle''.  It is possible for a temporarily merged sub-halo
to re-emerge from its host halo  at a later snapshot because it hasn't
actually dissolved  or merged completely,  but only because  it wasn't
detected  by the  clump finder  as a  separate density  peak.  Such  a
situation  is illustrated  in Figure  \ref{fig:jumper-demo}. In  these
cases,  orphan  particles  are  used   to  establish  a  link  between
progenitor  and  descendant   clumps  across  non-adjacent  snapshots.
By  default, \texttt{ACACIA} will track orphans until the end of the 
simulation, and orphans are only removed after they have indeed established
a link between a progenitor and descendant and thus have served their
purpose. Nonetheless, the current implementation 
offers the option to remove orphan particles after a user defined number
of snapshots has passed. Keeping track of orphan particles indefinitely
might lead to misidentifications of progenitor-descendant pairs and 
therefore to wrong formation histories. Our analysis shows however that 
matches between progenitor-descendant pairs over an interval greater 
than 10 snapshots are quite rare, so we expect this type of 
misidentifications to be a negligible issue. 

Additionally, orphan particles also serve a second  purpose. If we also
want to produce a mock galaxy catalogue on-the-fly using a dark matter
only simulation, the orphan particles are also used to track orphan
\emph{galaxies}. Those are  galaxies that  don't have  an associated
dark  matter  clump  any  longer  because  of  numerical  overmerging.
If we interpret orphan particles as orphan galaxies, there could be 
additional reasons to consider stopping tracking them. For example, the
effects of dynamical friction makes them fall towards the central 
galaxies. Once the orphan galaxies have lost enough energy, they may 
find themselves in close proximity to the central galaxies, even below 
the resolution limit. In these cases, it makes little sense to keep 
track of these orphans as individual galaxies. They should rather be 
regarded as merged into the central galaxy, and for that reason removed 
from the list of tracked orphans. Given that the model we employ
doesn't provide us with the galaxy radii, this approach requires some 
form of galaxy-galaxy merging cross-sections to compute the probability
of a collision between galaxies that will result in a galaxy merger. 
A different approach that other models use is to estimate the time for 
orphan galaxies to merge into the parent structure. This estimate could 
be e.g. the dynamical friction time (as is done in \citet{Moster}), or 
the fitting formula for the merger timescale of galaxies in cold dark 
matter models by \citet{merger_timescales}. These physically motivated
approaches to remove orphans will be the subject of future work.



%From a technical viewpoint, one clear bottleneck in the current merger tree algorithm is the requirement to write progenitor particles and data to file and read them back in and sort them out at a later snapshot. 
%A possible improvement would be to track which particles left each task's domain and which particles entered in between two snapshots. The progenitor particles would still be read and written to and from files, but it would minimise the sorting part of the algorithm where each MPI task figures out which tracer particles it contains.
%Another option would be to change the amount of data each MPI task needs to read in. 
%The maximal velocity of any particle in the time interval between two snapshots should be traced. This way, once the simulation advances to the next snapshot, it would be possible to estimate the maximal distance any particle could've travelled.
%Provided every MPI task has knowledge on how the entire computational domain is split between tasks, it could skip reading in data written by tasks where no particle currently in this task's domain could have come from.

%In this work, two mock galaxy catalogues were generated from dark matter simulations using the  \cite{Behroozi} SMHM relation and the merger trees obtained on the fly with \texttt{ACACIA}.
%The obtained stellar mass functions (Figure \ref{fig:smf}) demonstrate that the \cite{Behroozi} SMHM relation can be used with the built-in clump finder \phew.
%Finally, we demonstrate the influence of the merger trees on the mock galaxy catalogues by computing two-point correlation functions of the galaxies, shown in Figure \ref{fig:correlations}.
%The influence of the merger tree on the galaxy catalogue is twofold: 
%Firstly, the sub-halo masses used to obtain the satellite galaxies are peak masses during the sub-halo's formation history, and secondly, no orphan galaxies can be tracked without \texttt{ACACIA}.
%The correlation functions have been evaluated both with and without orphan galaxies.
%When orphan galaxies are included, much better results are obtained.
%Given that the mock galaxy catalogues in this work were created using simulations with relatively low spatial and mass resolution of $512^3$ particles in boxes of 69 and 100 comoving Mpc each, the obtained stellar mass functions and correlation functions show good agreement with observed stellar mass functions.
%By comparing the results of the two simulations it can be expected that a higher spatial resolution should improve the clustering statistics, and together with a higher mass resolution the stellar mass functions of central galaxies should also improve.
%Thereby we demonstrate that our new merger tree builder and the generation of mock galaxy catalogues from DMO simulations can be successfully performed on the fly with \ramses.

Finally, as a  proof of concept and using the  known formation history
of  dark matter  clumps from  the merger  trees and  a widely  adopted
stellar-mass-to-halo-mass  relation  \citep{Behroozi}, we  generate  a
mock  galaxy  catalogue  from  a dark  matter  only  simulation.   The
influence of the merger trees on  the quality of the galaxy catalogues
is twofold.   First, while the stellar-mass-to-halo-mass  relation can
be directly  applied to  central galaxies  associated to  main haloes,
using the  peak clump  mass for  sub-haloes is  a better  approach for
satellite galaxies.   The reason is  that tidal stripping  of galaxies
inside  a dark  matter halo  sets in  much later  than for  their host
sub-halo \citep{Nagai}.  Second, without properly keeping track of all
merging  events, no  orphan  galaxies  can be  traced,  nor can  their
stellar  mass  be   estimated  through  the  stellar-mass-to-halo-mass
relation  unless  it's  known  from   which  halo  the  orphan  galaxy
originated from, and what properties this halo had in the past.

To highlight  the impact of  the merger trees, we  compute observables
from our  mock galaxy catalogue,  both including and  excluding orphan
galaxies.  Specifically,   we  compute  the  stellar   mass  two-point
correlation  functions   and  radial  profiles  of   projected  number
densities and projected stellar mass densities in galaxy clusters. When orphan
galaxies are included in the analysis, we obtain correlation functions
and radial  profiles in  good agreement with  observations, validating
the different steps in our overall methodology.

The \ramses\  code is  publicly available and  can be  downloaded from
\url{https://bitbucket.org/rteyssie/ramses/}.  Instructions  on how to
use \texttt{ACACIA} and \texttt{PHEW} during a simulation can be found
under \url{https://bitbucket.org/rteyssie/ramses/wiki/Content}. 

%=================================================
\section*{Data Availability}
%=================================================

The data underlying this article will be shared on reasonable request 
to the corresponding author.

%=================================================
\section*{Acknowledgements}
%=================================================

MI would like  to thank B. Roukema for  helpful suggestions concerning
the history  of the application  of merger trees in  the astrophysical
context, S.   Avila for discussions  on details of the  Sussing Merger
Tree Comparison  project, and Y. Revaz for  his support in many  ways.  
This  work  was supported  by the  Swiss
National Supercomputing  Center (CSCS)  under projects s1006  and uzh5
and  by the  Swiss  National Science  Foundation  (SNF) under  project
72535 ``Multi-scale multi-physics models of galaxy formation''.  This
work        made        use        of        the        \textsc{numpy}
\citep{harrisArrayProgrammingNumPy2020}       and       \textsc{scipy}
\citep{virtanenSciPyFundamentalAlgorithms2020}  python  libraries  for
the      data      analysis      and      the      \textsc{matplotlib}
\citep{hunterMatplotlib2DGraphics2007}  python  library  for  plotting
tools.

       
       
       
       
       
%%%%%%%%%%%%%%%%%%%%%%%%%%%%%%%%%%%%%%%%%%%%%%%%%%

%%%%%%%%%%%%%%%%%%%% REFERENCES %%%%%%%%%%%%%%%%%%

    % The best way to enter references is to use BibTeX:
    
    \bibliographystyle{mnras}
    \bibliography{references} % if your bibtex file is called references.bib
    
    \appendix
   	%===========================================================================================
\section{Detailed Description of the Merger Tree algorithm}\label{app:detailed_mergertree}
%===========================================================================================


The merger tree code starts once clumps have been identified and the particle unbinding is completed.
The algorithm is designed to work in parallel on distributed memory architectures and makes use of the MPI standard.
When \ramses\ is executed, each processing unit involved in the execution is typically referred to by a unique index, 
called the ``MPI rank'', and given a unique domain, i.e. part of the volume of the simulated space, to work on. 
In what follows, we assume that the code is being executed in parallel over multiple MPI ranks.
The code then proceeds as follows:

\begin{enumerate}
%	
	\item Write the current clump data to file which is to be read in as progenitor data in the subsequent snapshot:
		For every clump, identify up to $n_{mb}$ tracer particles with minimal binding energy.
		If a clump consists of less than $n_{mb}$ particles, then take the maximally available number of particles.
		Then write the tracer particles of all ranks into a single shared file. 
		This file will be read in by every rank in once the next snapshot is being processed.
		If past merged progenitors exist, each rank writes these to (a different) shared file for later use.

%		
%	
	\item Every rank reads in the progenitor data from the shared file of the previous snapshot.
	
	\item The read in progenitor data is processed:
		First we find which progenitor tracer particles are on each rank's domain by comparing the tracer particles' unique ID to the IDs of the particles currently in this rank's domain.
		Each rank needs to know which tracer particles are currently on its own domain.	
		Then we find and communicate globally which rank is the ``owner'' of which progenitor (and past merged progenitor): 
		The owner of any progenitor is defined as the rank which has the most strongly bound particle of that progenitor within its domain.
		This particle is referred to as the ``galaxy particle'' of this progenitor.
		Each rank henceforth only keeps track of the tracer particles that are on its domain.
		The rest are removed from memory.

	
	\item Find links between progenitors and descendants, that is find ``which tracer particle ended up where'':
	
		After clump finding, the clump to which any particle belongs is known, and after reading in progenitor data, the progenitor clump 
		to which any tracer particle belonged to is known.
		Each rank now loops through all its local tracer particles.
		Using these two informations (in which clump the particle was and in which clump the particle is now) for every tracer particle, 
		all descendant candidates for all progenitors are found and stored in a sparse matrix, where the rows of the matrix correspond 
		to progenitors and the columns are the descendants.
		The exact number of particle matches between a progenitor-descendant candidate pair is kept.
%		For example: let $n_{mb}=200$. For the progenitor with ID 1, a possible result would be to find 50 particles in descendant with ID 2, 120 particles in descendant with ID 7, 10 particles in descendant 3 and 20 particles that aren't in a clump at the current snapshot.
		
		With the sparse matrices populated, they are now communicated across ranks where they are needed. 
		First every rank that has data on progenitors that it doesn't own itself sends this data (specifically, the sparse matrix data) to the owner of that progenitor.
		The owners then gather and sum up all the matches found in the previous linking step for the progenitors that they own and then send them 
		back to any rank that has at least one particle of that progenitor on their domain.
		(These are the same ranks that sent data to the owner of the progenitor in the first place.)
		
		After communications are done, a transverse sparse matrix is created, where the rows are descendants and the columns are progenitors.
		These matrices will be used to loop through progenitor or descendant candidates.
		

	
	\item Make trees:
		We first obtain an initial guess for the main progenitors of every descendant and for the main descendant of every progenitor by
		finding the candidate that maximises the merit function given by Equation~\eqref{eq:merit}.
		
		Then we loop to establish matches:
		
		A main progenitor-descendant pair is established when the main progenitor of a descendant is the main descendant of said progenitor, or in pseudocode:
		\begin{verbatim}
		match = (main_prog(idesc)==iprog) && 
        (main_desc(iprog)==idesc)
		\end{verbatim}
		
		While there are still descendants without a match and still progenitor candidates left for these descendants:
		
		\begin{itemize}
			
			\item For progenitors without a match: Loop through all descendant candidates. 
					If you find a match, stop there and mark this descendant candidate as the main descendant for this progenitor.
			
			\item Then for all descendants still without a match: Switch to the next best progenitor candidate as current best guess.
			
		\end{itemize}
		
		The loop ends either when all descendants have a match, or if descendants run out of candidates.
		If a progenitor hasn't found a match at this point, we assume that it merged into its best descendant candidate, i.e. the one that maximises the merit function.
	  The merged progenitors are added to the list of past merged progenitors by adding their ``galaxy particle'' to the list.
		
		If there are descendants that still have no main progenitor, we now try finding a progenitor from an older, non-consecutive snapshot.
		Past merged progenitors are tracked by one particle, their former ``galaxy particle'', which we now refer to as the ``orphan particle''.
		All particles of the descendant under investigation are checked for being an orphan particle of a past merged progenitor.
		The most strongly bound orphan particle will be considered the main progenitor of the descendant under consideration.
		If a match is found, the past merged progenitor is removed from the list of past merged progenitors.
		
		Finally, descendants that still haven't found a progenitor at this point are deemed to be newly formed.
		
	\item The results are written to file.
	
	

	
\end{enumerate}

















%========================================================================================================
\section{Tree Statistics Using \citet{SUSSING_HALOFINDER} Selection Criteria}\label{app:performance_comparison}
%========================================================================================================



\begin{table}
\centering
\caption{
	Comparison of simulation and evaluation parameters used in this work and of A14, where the parameters of the latter have been converted using $h = 0.704$.
	$m_m$ is the mass threshold for main haloes, $m_s$ is the mass threshold for sub-haloes.
	\label{tab:parameter-comparison}
}
	\begin{tabular}[c]{l l l}
													&	This work		&	A14 \\
		\hline
		particle mass	[$10^9 \msol$]				&	$1.55$			& $1.32$						\\
		particles used								& 	$256^3$			& $270^3$ 						\\
		box size [Mpc]								& 	$88.6$			& $88.8$						\\
		snapshots until $z = 0$						&  	67				& 62							\\				
		$m_m$ [$10^{12} \msol$]						&	$1.51$			& $1.12$ - $1.37$				\\
		$m_s$ [$10^{11} \msol$]						&	$3.91$			& $4.26$ - $9.72$				\\
		\hline
	\end{tabular}
\end{table}


\begin{figure}
	\centering
	\includegraphics[width=.9\linewidth, keepaspectratio]{images/tree-statistics-sussing-threshold/branching-ratio-ntrace.png}\\%
	\caption{
		Histogram of the number of direct progenitors for all clumps from $z = 0$ to $z = 2$ for $n_{mb} = 100$ and $1000$ tracer particles per clump.
		The histogram is normalized by the total number of events found.
	}%
	\label{fig:sussing-branching-ratio}
\end{figure}

\begin{figure}
	\centering
	\includegraphics[width=.95\linewidth, keepaspectratio]{images/tree-statistics-sussing-threshold/main-branch-lenghts-all-bins-ntrace.png}%
	\caption{
		Histograms of the length of the main branch. 
		The left plot shows the length of the 1000 most massive haloes at $z = 0$, the right plot shows the length of the 200 most massive sub-haloes for $n_{mb} = 100$ and $1000$ tracer particles per clump.
	}%
	\label{fig:sussing-branch-lengths}
\end{figure}

\begin{figure*}
	\centering
	\includegraphics[width=\textwidth, keepaspectratio]{images/tree-statistics-sussing-threshold/mass_growth-ntrace.png}%
	\caption{
		Logarithmic mass growth for haloes and sub-haloes satisfying the mass thresholds.
		Group $A$ contains clumps that are either haloes or sub-haloes in consecutive snapshots $k$ and $k+1$ with masses $m \geq m_{m}$.
		Group $B$ contains clumps that are only haloes in two consecutive snapshots with mass above $m_{m}$, group $C$ contains only clumps that were sub-haloes in two consecutive snapshots with mass greater than $m_{s}$.
		The histogram is normalized by the total number of events found for group $A$.
	}%
	\label{fig:sussing-mass-growth}
\end{figure*}

\begin{figure*}
	\centering
	\includegraphics[width=\textwidth, keepaspectratio]{images/tree-statistics-sussing-threshold/mass_fluctuations-ntrace.png}\\%
	\caption{
		Histogram of mass growth fluctuations for haloes and sub-haloes satisfying the mass thresholds.
		Group $A$ contains clumps that are either haloes or sub-haloes in three consecutive snapshots with masses $m \geq m_{m}$.
		Group $B$ contains clumps that are only haloes in three consecutive snapshots with mass above $m_{m}$, group $C$ contains only clumps that were sub-haloes in three consecutive snapshots with mass greater than $m_{s}$.
		The histogram is normalized by the total number of events found for group $A$.
	}%
	\label{fig:sussing-mass-fluct}
\end{figure*}




In this section, the merger tree statistics introduced in Section \ref{chap:tests} when following the selection criteria that are used in \citet{SUSSING_HALOFINDER} (A14 from here on) are presented.
Ideally, \texttt{ACACIA} should be tested on the same datasets and halo catalogues used in the Comparison Project to enable a direct comparison to the performance of other merger tree codes. 
However, since \texttt{ACACIA} was designed to work on the fly, using it as a post-processing utility would defeat its purpose.
Furthermore, \texttt{ACACIA} is not necessarily compatible with other definitions of haloes and sub-haloes.
But most importantly, we also want to demonstrate that the halo finder \phew\ can be used to produce reliable merger trees. 
So instead, the tests are performed on our own datasets and halo catalogues, which are described in section \ref{chap:testing_methods}.
A comparison of the used parameters of our simulations and the ones used in A14 is given in Table \ref{tab:parameter-comparison}.
In the following, the results for $n_{mb} = 100$ and $n_{mb} = 1000$ are shown.
Like before, when the influence of the number of tracer particles was investigated, the \sad\ parameter and the \exc\ mass definition were used.


The difference to the results presented in Section \ref{chap:tests} is that the mass thresholds are set such that only the 1000 most massive main haloes and only the 200 most massive sub-haloes at $z = 0$ are included.
This gives effective mass threshold $m_{m} = 1.51 \times 10^{12} \msol$ and $m_{s} = 3.9 \times 10^{11}\msol$, which are on one hand comparable to the mass thresholds applied in A14 (Table \ref{tab:parameter-comparison}), but already show differences in the resulting halo catalogue.
\phew\ finds a higher mass threshold for main haloes, but a lower mass threshold for sub-haloes.
This is consistent with the fact that the \sad\ parameter was used:
Unbound particles are passed on to substructure that is higher up in the hierarchy, and the unbinding is repeated until the top level, which are the main haloes, is reached.
The more strict unbinding criterion tends to assign more particles to the main haloes and remove them from sub-haloes, which is reflected in the mass thresholds.
Indeed, using the \nosad\ parameter instead leads to $m_m = 1.39 \times 10^{12}\msol$ and $m_s = 2.00 \times 10^{12}\msol$.


The length of the main branches for haloes and sub-haloes individually are shown in Figure \ref{fig:sussing-branch-lengths}.
For the halo population, two noticeable differences compared to Figure 3 of A14 appear:
%
\begin{enumerate}
	\item \texttt{ACACIA} finds some main haloes with short (< 10) main branches.
	In A14, his only happens for the \texttt{JMerge} tree maker, \texttt{TreeMaker} for \texttt{Rockstar} haloes, and \texttt{SubLink} for \texttt{AHF} haloes. 
	\item Most clumps satisfying the mass thresholds have very large main branch lengths that are in a narrow range ($\sim 10$) of snapshots, while the high main branch length distribution found in A14 is much wider (> 20).
\end{enumerate}
%
This indicates that on one hand, \texttt{ACACIA} probably makes more misidentifications, hence the short main branches, but simultaneously is able to track clumps to higher redshifts.

In Figure \ref{fig:sussing-branching-ratio} the number of direct progenitors for all clumps between $z = 0$ and $z = 2$ are shown.
Comparing to Figure 5 of A14, \texttt{ACACIA} gives very comparable results:
$\sim 10^{-1}$ haloes have no direct progenitor, almost all have one, and the distribution follows an exponential decay with the maximal number of direct progenitors lying around 20-25.
Many tree makers and halo finders in A14 exhibit the same kind of behaviour, particularly so for the \texttt{AHF}, \texttt{Subfind}, and \texttt{Rockstar all} halo finders in Figure 5 of A14.



For the logarithmic mass growth (Figure \ref{fig:sussing-mass-growth}) and the mass growth fluctuations (Figure \ref{fig:sussing-mass-fluct}), the statistics are separated into three groups.
Group $A$ contains clumps that are either haloes or sub-haloes in \emph{consecutive} snapshots $k$ and $k+1$ with masses $m \geq m_{m}$.
Group $B$ contains clumps that are exclusively main haloes in two consecutive snapshots with mass above $m_{m}$, group $C$ contains only clumps that were sub-haloes in two consecutive snapshots with mass greater than $m_{s}$.
Except for the branching ratio statistic, we follow clumps of the $z = 0$ snapshot along the main branch only.

The logarithmic mass growth resulting from \texttt{ACACIA} follows the general trend that the tree makers in A14 exhibit too.
The growth for groups $A$ and $B$ increases steadily and peaks around $\beta_M \sim 0.5$, where the peak is $\sim 10^{-2}$.
For $n_{mb} = 100$, the extreme mass loss with $\beta_M = -1$ increases for group $A$, which is an undesirable property, but is also exhibited by \texttt{Sublink} in A14.
For $n_{mb} = 1000$, it drops below $10^{-3}$ ($10^{-4}$ for group $B$), which is comparable behaviour to \texttt{MergerTree}, \texttt{Sublink}, and \texttt{VELOCIraptor}, particularly so in combination with \texttt{AHF} and \texttt{Subfind} halo finders.
Group $C$, containing only mass growths of clumps that have been sub-haloes in two consecutive snapshots, shows a distribution peaking around extreme mass growths $\beta_M \rightarrow \pm 1$ at $\sim 5 \times 10^{-3}$, which can again be seen in A14 for almost all tree makers, albeit not for all halo finders.
More noticeably, almost no sub-haloes are found with $-0.5 < \beta_M < 0.5$ with \phew\ and \texttt{ACACIA} in Figure \ref{fig:sussing-mass-growth}.
This is most likely due to the fact that once a halo is merged into another, it quickly looses its outer mass due to to the strict unbinding method used here.
%Since only clumps that are identified as sub-haloes are included in group $C$, the statistics during the clump's lifetime as a halo are not included.

The mass growth fluctuations (Figure \ref{fig:sussing-mass-fluct}) of \texttt{ACACIA} share the general trend with the ones from Figure 8 in A14, in that they peak around $\xi_M = 0$ and decrease outwards towards $\xi_M = \pm 1$.
In A14, in all cases groups $A$ and $B$ peak just below $10^{-1}$, while our results peak just above $10^{-2}$.
However, similarly to the results of e.g. \texttt{Sublink}, \texttt{TreeMaker}, and \texttt{VELOCIraptor} with the \texttt{AHF} or \texttt{Subfind} halo finders, the distribution around $\xi_M \sim \pm 0.5$ drops to $\sim 5 \times 10^{-3}$, and then continues dropping below $10^{-4} - 10^{-3}$ at $\xi_M \sim \pm 1$.
For $n_{mb} = 100$, the group $A$ distribution rises again for $\xi_M \sim 1$, as it does for \texttt{TreeMaker} and \texttt{Sublink} tree makers with \texttt{AHF}.
Group $B$ shows a steeper drop around the extreme values $\xi_M \sim \pm 1$ compared to group $A$, dropping below $10^{-4}$ at these values, similarly to the behaviour of many tree makers and halo finders in A14.
The sub-halo group $C$ of this work shows three main peaks, around $-1$, $0$, and $1$.
These peaks also appear in the A14 results.
However, the peaks at the extreme values in A14 are lower than the ones of this work, while the peaks around $0$ is higher.
The missing values around $\xi_M \sim \pm 0.5$ that were also seen in the mass growth in Figure \ref{fig:sussing-mass-growth} remain unsurprisingly.
Similar distributions are obtained by \texttt{Sublink} and \texttt{VELOCIraptor} in combination with the \texttt{AHF} halo finder.




In summary:

\begin{enumerate}
	\item \texttt{ACACIA} finds more massive haloes with short main branch lengths, but the main branch length distribution peaks at very high numbers and is narrower
	\item The distribution of the number of direct progenitors for all haloes within $0 < z < 2$ is consistent with the results from A14
	\item The logarithmic mass growth is similar to what is obtained with some other halo finders and tree makers, in particular the \texttt{Sublink} and \texttt{VELOCIraptor} tree makers in combination with the \texttt{AHF} and \texttt{Subfind} halo finders. 
	\item Apart of the higher peak value of the distribution in A14, the mass growth fluctuations are similar to the results of the \texttt{Sublink} and \texttt{VELOCIraptor} tree makers in combination with the \texttt{AHF} halo finder.
\end{enumerate}

Hence we conclude that our tree maker gives comparable results with respect to other state of the art merger tree and halo finding codes.

%%%%%%%%%%%%%%%%%%%%%%%%%%%%%%%%%%%%%%%%%%%%%%%%%%
   	
  	
	% Don't change these lines
	\bsp	% typesetting comment
	\label{lastpage}
\end{document}