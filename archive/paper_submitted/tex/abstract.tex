\begin{abstract}
  The implementation of \texttt{ACACIA}, a new algorithm to generate
  dark matter halo merger trees with the Adaptive Mesh Refinement
  (AMR) code \ramses, is presented.  The algorithm is fully parallel
  and based on the Message Passing Interface (MPI). As opposed to most
  available merger tree tools, it works on the fly during the course
  of the N body simulation.  It can track dark matter substructures
  individually using the index of the most bound particle in the
  clump.  Once a halo (or a sub-halo) merges into another one, the
  algorithm still tracks it through the last identified most bound
  particle in the clump, allowing to check at later snapshots whether
  the merging event was definitive, or whether it was only temporary,
  with the clump only traversing another one.  The same technique can
  be used to track orphan galaxies that are not assigned to a parent
  clump anymore because the clump dissolved due to numerical
  over-merging.  We study in detail the impact of various parameters
  on the resulting halo catalogues and corresponding merger histories.
  We then compare the performance of our method using standard
  validation diagnostics, demonstrating that we reach a quality
  similar to the best available and commonly used merger tree tools.
  As a proof of concept, we use our merger tree algorithm together
  with a parametrised stellar-mass-to-halo-mass relation and generate
  a mock galaxy catalogue that shows good agreement with observational
  data.
\end{abstract}
